\documentclass{article}


\usepackage{booktabs}
\usepackage{float}
\usepackage{graphicx}
\usepackage[utf8]{inputenc}
\usepackage{caption}
\usepackage{subcaption}
\usepackage[a4paper, margin=0.5in]{geometry}
\usepackage{titlesec}
\usepackage{setspace}
\usepackage{ragged2e}
\usepackage{fancyhdr}





\titleformat{\section}{\Large\bfseries}{\thesection}{1em}{}
\titleformat{\subsection}{\large\bfseries}{\thesubsection}{1em}{}


\captionsetup{font=small, labelfont=bf}


\justifying
\setlength{\parindent}{15pt}
\raggedbottom


\pagestyle{fancy}
\fancyhf{}
\fancyfoot[C]{\thepage}
\renewcommand{\headrulewidth}{0pt}


\begin{document}

\title{Apple Stock AR-Process Analysis and Multistep Forecasting}
\author{Lennart Epp}
\date{\today}

\maketitle


\pagenumbering{gobble}


\tableofcontents
\listoffigures
\listoftables

\newpage
\pagenumbering{arabic}

\section{Introduction}

In this project, Apple stock data is analyzed using time series econometrics methods. This
paper is structured as follows: first, I present the Top 3 best-fitting AR(p) models for approximating
i.e for one step forecasting of the Apple stock data. So for this forecast/approximation, only values
of the original data namley the close price of the apple stock are used as in input for the forecast.\\
Then, I proceed with an analysis to what extent it is possible to fit an AR(p) process on Apple.
For this I checked if the differenced Apple data is stationary and after that I concern whether the
time series has long or short memory.\\
Lastly, I will present a multi-step forecast, where each forecasted value is used as input
for the next prediction. I will analyze why this approach fails to capture Apple stock dynamics
beyond a one-step forecast and discuss the overall feasibility of fitting an AR model to Apple stock data.

\section{Top AR(p) Approximations}

The following plots shows the top 3 AR model fits in the sense of the Akaike Information
Criterion. It also shows the residual plot of the top 3 AR models. The figure shows that
one-step forecasts closely follow the original data. However, the variance of the residuals
increases over time, suggesting that the model struggles to maintainforecast accuracy over
longer periods.

\begin{figure}[H]
    \centering
    \includegraphics[scale=1.8, width=\textwidth, trim=10 10 10 10, clip]
    {../bld/plots/top_ar_models_plot.pdf}
    \caption{Comparison of the top-performing AR models.}
    \label{fig:top_ar_models}
\end{figure}

\noindent In the following table, you see the metrics of the best AR(P) processes in terms of their AIC.
First, notice that the AR(p) was fitted on the differenced close price since the P-Value of
the Augmented Dickey-Fuller test suggested differencing, indicating that the original close
price is likely not stationary.\\
Therefore, the AR coefficients had to be integrated to approximate the original time series,
which could lead to accumulated errors. So given the AIC as you can see in the table, the AR(1) process fitted
Apple best. In total I tested p values up to 12.

\input{../bld/models/top_models.tex}

\section{Memory Analysis}

In this section, I check to what extent it is possible to fit an AR model on the differenced data.
Therefore, I first used the ADF test to check if the differenced close price is stationary.
The results are shown in the following table, indicating that the differenced series is likely
stationary. This is a necessary prerequest for fitting AR models.

\input{../bld/memory/diff_close_stat_test.tex}


\noindent After confirming stationarity, the next critical question is whether the differenced time series
exhibits short or long memory. Therefore I computed the Autocorrelation Function of the time series. As
you can see in the following plot the ACF decreases over time but still has a few outlieres which
indicates that the time series has the characteristics of a process with a short memory,
but with potential components with a long memory.

\begin{figure}[H]
    \centering
    \includegraphics[scale=1.2, width=\textwidth, trim=10 10 10 10, clip]
    {../bld/plots/acf_plot.pdf}
    \caption{Autocorrelation Function (ACF) of the differenced time series with 95\% confidence
    bands.}
    \label{fig:acf_plot}
\end{figure}


\noindent Since the differenced close price is likely stationary, but the ACF indictaes some long run effects
I also cumpted the hurst coefficient which has an value of approximately  .052 which indicates
that the time series shows characteristics of a process with almost random behavior, as the Hurst coefficient
close to 0.5 indicates the absence of strong long-term dependencies. However, since the ACF indicates some
long-term effects, this result could indicate a mixture of short-term autocorrelations with occasional
persistence.


\input{../bld/memory/hurst_exponent.tex}

\newpage

\section{Multistep Forecast}

Although the differenced close price was found to be stationary, the presence of a Hurst
coefficient of 0.52 and some significant autocorrelation function (ACF) values suggest that
the series retains some degree of long memory.

\noindent AR models are designed to capture short-term dependencies and assume that the impact of past
values decays rapidly. However, in a long-memory process, dependencies persist for a longer
time, meaning that an AR(p) model may fail to account for the full structure of the series
beyond a few steps ahead.

\noindent In a one-step-ahead forecast, the AR model predicts the next value based solely on observed
historical data. In a multi-step forecast, each predicted value is used as input for the next
prediction. This recursive approach leads to error accumulation.

\noindent The following figure illustrates that the AR model fails to capture the long-term structure
of Apple stock price movements when applied to multi-step forecasting. This is due to error accumulation
and the model's inability to account for evolving market dynamics.


\begin{figure}[H]
    \centering
    \includegraphics[scale=1.8, width=\textwidth, trim=10 10 10 10, clip]
    {../bld/plots/multistep_forecast.pdf}
    \caption{Multi-step forecast for Apple stock price.}
    \label{fig:apple_forecast}
\end{figure}

\section{Conclusion}

To conclude my analysis i went through the following steps:\\
First, the stationarity analysis, conducted using the Augmented Dickey-Fuller (ADF) test,
indicated that the original close price series was non-stationary, requiring differencing to
achieve stationarity.\\
Second, the evaluation of different AR(p) models based on the Akaike Information Criterion
(AIC) revealed that an AR(1) model provided the best fit among the examined options.\\
Third, the study investigated the memory characteristics of the time series by computing the
Hurst exponent and analyzing the autocorrelation function (ACF). The results suggested that
the differenced time series exhibited short-memory behavior.\\
Finally, the limitations of AR models for multi-step forecasting were assessed. While the AR(1)
model performs well for short-term forecasts, its accuracy deteriorates over multiple steps
due to error propagation and the inability to capture long-term dependencies.

\end{document}
