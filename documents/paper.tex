\documentclass{article}


\usepackage{booktabs}
\usepackage{float}
\usepackage{graphicx}
\usepackage[utf8]{inputenc}
\usepackage{caption}
\usepackage{subcaption}
\usepackage[a4paper, margin=0.5in]{geometry}
\usepackage{titlesec}
\usepackage{setspace}
\usepackage{ragged2e}
\usepackage{fancyhdr}





\titleformat{\section}{\Large\bfseries}{\thesection}{1em}{}
\titleformat{\subsection}{\large\bfseries}{\thesubsection}{1em}{}


\captionsetup{font=small, labelfont=bf}


\justifying
\setlength{\parindent}{15pt}
\raggedbottom


\pagestyle{fancy}
\fancyhf{}
\fancyfoot[C]{\thepage}
\renewcommand{\headrulewidth}{0pt}


\begin{document}

\title{Autoregressive Model Analysis and Multistep Forecast of Apple Stock Data}
\author{Lennart Lülsdorf}
\date{\today}

\maketitle


\pagenumbering{gobble}


\tableofcontents
\listoffigures
\listoftables

\newpage
\pagenumbering{arabic}

\section{Introduction}

In this project, Apple stock data is analyzed using time series econometrics methods. This
project is structured as follows: first, the three best autoregressive AR(p) models for approximating
i.e. for one step forecasting of the Apple stock data are presented. Therefore, for this approximation, only values
of the original data, namley the close price of the apple stock, are used as an input for the forecast.\\
This was followed by an analysis, to what extent it is possible to fit an AR(p) process on Apple.
To do this, the stationarity of the differenced Apple data was examined and also whether the time series has long or short memory.\\
Finally, a multi-step forecast is presented, where each forecasted value is used as an input
for the next prediction. The failure of this approach to capture Apple stock dynamics beyond a one-step forecast
will be analyzed and a discussion of the overall feasibility to fit an AR model to Apple stock data completes the project.

\section{Top AR(p) Approximations}

In the first graph of Figure 1 the three best AR model fits in the sense of the Akaike Information
Criterion (AIC) are shown. The second graph visualizes the residual plot of these AR models. Overall, the figure displays
that one-step forecasts closely follow the original data. However, the variance of the residuals
increases over time, suggesting that the model struggles to maintain forecast accuracy over
longer periods.

\begin{figure}[H]
    \centering
    \includegraphics[scale=1.8, width=\textwidth, trim=10 10 10 10, clip]
    {../bld/plots/top_ar_models_plot.pdf}
    \caption{Comparison of the top-performing AR models}
    \label{fig:top_ar_models}
\end{figure}

\noindent Table 1 contains the metrics of the best AR(p) processes in terms of their AIC.
Notice that the AR(p) model was fitted on the differenced time series of the close price from Apple stock data,
since the p-value of the Augmented Dickey-Fuller (ADF) test showed that the original close price time series
is nonstationary.\\
Therefore, the AR coefficients had to be integrated, to approximate the original time series,
which increases the probability of accumulated errors. So, given the AIC the AR(1) process fitted
Apple best (Table 1). In total, values for p from 1 to 12 were tested.

\input{../bld/models/top_models.tex}

\section{Memory Analysis}

This section addresses the question to what extent it is possible to fit an AR model on the differenced data.
Therefore, the ADF test was used to check if the differenced close price was stationary.
The results are shown in Table 2, indicating that the differenced series is likely
stationary, which is a necessary prerequest for fitting AR models.

\input{../bld/memory/diff_close_stat_test.tex}


\noindent After confirming stationarity, the next critical question is whether the differenced time series
exhibits short or long memory. Therefore, the Autocorrelation Function (ACF) of the time series was computed. Visualized
in Figure 2, the ACF decreases over time, which indicates the characteristics of a process with short memory.
However, a few outlieres visualize a potential of a time series with long memory.


\begin{figure}[H]
    \centering
    \includegraphics[scale=1.2, width=\textwidth, trim=10 10 10 10, clip]
    {../bld/plots/acf_plot.pdf}
    \caption{Autocorrelation Function (ACF) of the differenced time series with 95\% confidence
    bands}
    \label{fig:acf_plot}
\end{figure}


\noindent Since the differenced close price is likely stationary, but the ACF indictaes some long run effects,
the Hurst exponent was computed (Table 3). This exponent indicates the absence of strong long-term dependencies
with a value close to 0.5. For the Apple stock data a value of approximately  0.52 was computed. This demonstrates
that the time series shows characteristics of a process with almost random behavior. However, since the ACF indicates some
long-term effects (Figure 2), this result might be an indication for a mixture of short-term autocorrelations with occasional
persistence.


\input{../bld/memory/hurst_exponent.tex}

\newpage

\section{Multistep Forecast}

Although the differenced close price was found to be stationary, the presence of a Hurst
coefficient of 0.52 and some significant ACF values suggest that
the series retains some degree of long memory.

\noindent AR models are designed to capture short-term dependencies and assume that the impact of past
values decays rapidly. However, in a long-memory process, dependencies persist for a longer
time, meaning that an AR(p) model may fail to account for the full structure of the series
beyond a few steps ahead.

\noindent In an one-step-ahead forecast, the AR model predicts the next value based solely on observed
historical data. While in a multi-step forecast, each predicted value is used as an input for the next
prediction. This recursive approach leads to error accumulation.

\noindent Figure 3 illustrates that the AR model fails to capture the long-term structure
of Apple stock price movements when applied to multi-step forecasting. This is due to error accumulation
and the model's incapacity to account for evolving market dynamics.


\begin{figure}[H]
    \centering
    \includegraphics[scale=1.8, width=\textwidth, trim=10 10 10 10, clip]
    {../bld/plots/multistep_forecast.pdf}
    \caption{Multi-step forecast for Apple stock price}
    \label{fig:apple_forecast}
\end{figure}

\section{Conclusion}

The autoregressive model analysis of Apple stock data can be summarized in three steps:
First, the stationarity analysis, conducted using the Augmented Dickey-Fuller (ADF) test,
indicated that the original close price series was nonstationary, requiring differencing to
achieve stationarity. Second, the evaluation of different AR(p) models based on the
AIC revealed that an AR(1) model provided the best fit among the examined options.
Third, the study investigated the memory characteristics of the time series by computing the
Hurst exponent and analyzing the autocorrelation function (ACF). The results suggested that
the differenced time series exhibited a mixture of short-term autocorrelations with occasional
persistence.\\
Finally, the limitations of AR models for multi-step forecasting were assessed. While the AR
models performs well for short-term forecasts, its accuracy deteriorates over multiple steps
due to error propagation and the inability to capture long-term dependencies.

\end{document}
